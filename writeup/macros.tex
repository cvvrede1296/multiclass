%legendremacros

\makeatletter
\def\legendre@dash#1#2{\hb@xt@#1{%
  \kern-#2\p@
  \cleaders\hbox{\kern.5\p@
    \vrule\@height.2\p@\@depth.2\p@\@width\p@
    \kern.5\p@}\hfil
  \kern-#2\p@
  }}
\def\@legendre#1#2#3#4#5{\mathopen{}\left(
  \sbox\z@{$\genfrac{}{}{0pt}{#1}{#3#4}{#3#5}$}%
  \dimen@=\wd\z@
  \kern-\p@\vcenter{\box0}\kern-\dimen@\vcenter{\legendre@dash\dimen@{#2}}\kern-\p@
  \right)\mathclose{}}
\newcommand\legendre[2]{\mathchoice
  {\@legendre{0}{1}{}{#1}{#2}}
  {\@legendre{1}{.5}{\vphantom{1}}{#1}{#2}}
  {\@legendre{2}{0}{\vphantom{1}}{#1}{#2}}
  {\@legendre{3}{0}{\vphantom{1}}{#1}{#2}}
}
\def\dlegendre{\@legendre{0}{1}{}}
\def\tlegendre{\@legendre{1}{0.5}{\vphantom{1}}}
\makeatother

%short hands because we cannot decide on notation
\newcommand{\norm}[2]{N_{#1:#2}}

\def\setof#1{\mathord{\left\lbrace{#1}\right\rbrace}}
\def\floor#1{\mathord{\left\lfloor{#1}\right\rfloor}}
\def\ceil#1{\mathord{\left\lceil{#1}\right\rceil}}
\def\abs#1{\mathord{\left\vert{#1}\right\vert}}

% algorithm names:
\def\goodprime{\operatorname{GoodPrime}\nolimits}
\def\twistedsquareroot{\operatorname{TwistedSquareRoot}\nolimits}
\def\squareroot{\operatorname{SquareRoot}\nolimits}
\def\unitsfunction{\operatorname{Units}\nolimits}
\def\enoughcharacters{\operatorname{EnoughCharacters}\nolimits}

\def\fieldn{\QQ(\sqLprime{1},\ldots,\sqLprime{n})}
\def\fieldbeforen{\QQ(\sqLprime{1},\ldots,\sqLprime{n-1})}
\def\ringn{\ZZ[\sqLprime{1},\ldots,\sqLprime{n}]}
\def\ringm{\ZZ[\sqLprime{1},\ldots,\sqLprime{m}]}
\def\ringbeforem{\ZZ[\sqLprime{1},\ldots,\sqLprime{m-1}]}
\def\ringbeforen{\ZZ[\sqLprime{1},\ldots,\sqLprime{n-1}]}
\def\ringofintegers{\calO}

% letters
\newcommand{\ZZ}{\mathbb{Z}}
\newcommand{\units}[1]{\calO_{#1}^\times}
\newcommand{\unitssq}[1]{(\calO_{#1}^\times)^2}
%\newcommand{\Zmod}[1]{\mathbb{Z}/#1\mathbb{Z}}
\newcommand{\Zmod}[1]{\mathbb{Z}/#1}
\newcommand{\RR}{\mathbb{R}}
\newcommand{\FF}{\mathbb{F}}
\newcommand{\QQ}{\mathbb{Q}}
\newcommand{\CC}{\mathbb{C}}
\newcommand{\dual}{\vee}
\newcommand{\calK}{\mathcal{K}}
\newcommand{\calO}{\mathcal{O}}
\newcommand{\fp}{\mathfrak{p}}
\newcommand{\fP}{\mathfrak{P}}
\newcommand{\I}{\mathcal{I}}
\newcommand{\Isub}[1]{\mathcal{I}_{#1}}
\newcommand{\J}{\mathcal{J}}
\newcommand{\Rorder}{\mathcal{R}}
\newcommand{\Sorder}{\mathcal{S}}
\newcommand{\LL}{L}
\newcommand{\UL}{U_L}
\newcommand{\K}[1]{K_{#1}}
\newcommand{\Q}[1]{Q_{#1}}
\newcommand{\defined}{T}

\newcommand{\qunit}[1]{\varepsilon_{#1}}
\newcommand{\sig}[1]{\sigma_{#1}}
\newcommand{\qchar}[1]{\chi_{#1}}
\newcommand{\uroot}[1]{\mu(#1)}

\newcommand{\rels}[1]{\mathcal{R}el(#1)}
\newcommand{\rel}[2]{\mathcal{R}_{#1}(#2)}

%subgroup of quadratic units of a MQ field
\newcommand{\Quadunits}[1]{Q_{#1}}

%multquad field dimensions and adjoined primes
\newcommand{\NN}{N}
\newcommand{\nn}{n}
\newcommand{\Lprime}[1]{d_{#1}}
\newcommand{\sqLprime}[1]{\sqrt{d_{#1}}}
\newcommand{\class}[1]{\mathcal{C}l_{#1}}
\newcommand{\pideals}[1]{\mathcal{P}(#1)}


%products of adjoined primes generating the quadratic subfields
\newcommand{\qprime}[1]{d_{#1}}
\newcommand{\sqprime}[1]{\sqrt{d_{#1}}}
\newcommand{\ind}{J}

%dual basis
\newcommand{\dualbasis}[1]{{#1}^{\vee}}


%basis of an ideal
\newcommand{\be}[1]{\omega_{#1}}

%operators
\DeclareMathOperator{\Span}{Span}
\DeclareMathOperator{\Gal}{Gal}
\DeclareMathOperator{\Ker}{Ker}
\DeclareMathOperator{\Hom}{Hom}
\DeclareMathOperator{\Log}{Log}
\DeclareMathOperator{\appLog}{ApproxLog}
\DeclareMathOperator{\sgn}{sgn}
\DeclareMathOperator{\id}{id}

%theorems and lemmas
%\newtheorem{definition}{Definition}
%\newtheorem{lemma}{Lemma}
\newtheorem{heuristic}[equation]{Heuristic}
%\newtheorem{proposition}{Proposition}
%\newtheorem{theorem}{Theorem}
%\newtheorem{corollary}{Corollary}

%\theoremstyle{plain}
\newtheorem{mythm}{Theorem}[section]
\newtheorem{mylma}[mythm]{Lemma}
\newtheorem{myprop}[mythm]{Proposition}
\newtheorem{mydef}[mythm]{Definition}
\newtheorem{mycor}[mythm]{Corollary}
\newtheorem{mycon}[mythm]{Conjecture}
