\section{The Algorithm}
\label{sec:alg}

\subsection{The algorithm}
\begin{algorithm}[t]
\DontPrintSemicolon
    \caption{{\rm MQClassGroup}($\LL$)}
    \label{alg:mqpip}
    \KwIn{Real multiquadratic field $\LL$}
    \KwResult{$\class \LL$}
        \If{$[\LL:\QQ] = 1$}{
            \Return \hbox{some error}
          }
        \If{$[\LL:\QQ] = 2$}{
            \Return {\rm QClassGroup}($\LL$)}
        $\sigma,\tau\gets \hbox{distinct non-identity automorphisms of $\LL$}$\;
        \For{$\ell \in \{\sigma,\tau,\sigma\tau\}$}{
            $K_\ell\gets \hbox{fixed field of $\ell$}$\;
            $\rels {\K \ell},\class {K_\ell} \leftarrow {\rm MQClassGroup}(\K \ell)$\;
                }
        $U \leftarrow \rels {\K \sigma} \cup \rels {\K \tau} \cup \sigma (\rels {\K {\sigma\tau}})$\;
        $\rels \LL \leftarrow {\rm findSquares}(U)$\;
        \Return ${\rels \LL, {\rm ClassGroup}(\rels \LL)}$\;
\end{algorithm}

\subsection{Heuristic running time}

\subsection{Open questions}
{\bf Q3: What happens to $\rels \LL$ if $\sigma$ is applied.\\
A3: Simple linear transformation of the relations (since we can compute $\sigma (\mathfrak{p})$ for all prime ideals)\\
\noindent
Q4: Are there precision issues?\\
A4: Since we are not dealing with embeddings, I don't see any issues.\\
\noindent
Q5: I still need to clarify how ${\rm findSquares}$ works.\\
A5:I think since all relations are essentially linear relations and squareroot of the 
relation in the quadratic fields, you only have to compute the characters for the relations in the 
quadratic fields and the you can just take linear combinations for the rest.
You get a theorem like Theorem 5.2 of~\cite{DBLP:conf/eurocrypt/BauchBVLV17}.\\
\noindent
Q6: This algorithm implies that the largest prime you need is the the largest prime of the largest quadratic field.
given the adjoined squareroots, the largest discriminant and therefor prime bound can be calculated.
This should be usable somehow.\\
\noindent
Q7: It also seems that for the all of one example I tried, the 
"The discriminant of a multiquadratic field is the product of the discriminant of its quadratic subfields".
This might be dependent on which squares are adjoined, but might be something to look into.\\
A7: This might be only for $1 \bmod 4$ primes? Or maybe not?\\
\noindent
Q8: Can the Biasse SAC 2017 preprocessing paper be used to make a class group precomputation to 
make~\cite{DBLP:conf/eurocrypt/BauchBVLV17} even faster? But this is only if everything works.\\
\noindent
Q9: There are still statements in Kubota~\cite{kubota1956} that I think could help, 
or at least should be referenced for completeness sake.
However, the German is too confusing for a novice.
}
